\documentclass[12pt, a4paper, oneside]{article}
\usepackage[UTF8]{ctex}
\usepackage{amsmath, amsthm, amssymb, bm, framed, graphicx, subcaption, mathrsfs, physics}
\usepackage{esvect}
\usepackage{xcolor}
\usepackage{tikz}
\usetikzlibrary{calc, arrows}
\usepackage{siunitx}
\usepackage[unicode]{hyperref}
\usepackage[capitalise]{cleveref}

\definecolor{shadecolor}{RGB}{241, 241, 255}

\sisetup{
  inter-unit-product = \ensuremath{{}\cdot{}},  % 单位间用点乘
  exponent-product  = \times,                  % 科学计数法用 × 而非 e
  group-separator   = {\,},                    % 千分位窄空格分隔
  separate-uncertainty                         % 不确定度用 ± 符号
}

\hypersetup{
  CJKbookmarks = true,
  bookmarksnumbered = true,
  pdfstartview = FitH,
  colorlinks = true,
  linkcolor = blue,
  anchorcolor = black,
  citecolor = green,
}

% 自定义公式引用格式
\crefname{equation}{Eq.}{Eq.}
\Crefname{equation}{Eq.}{Eq.}
\crefformat{equation}{Eq.(#2#1#3)}
\Crefformat{equation}{Eq.(#2#1#3)}

% 保持图和表格的默认引用格式
\renewcommand{\figurename}{Fig.}
\crefname{figure}{Fig.}{Figs.}
\renewcommand{\tablename}{Tab.}
\crefname{table}{Tab.}{Tabs.}

% % 定义 autoref 标签名
% \def\figureautorefname{Fig.}
% \def\tableautorefname{Tab.}
% \def\equationautorefname{Eq.}

\title{\textbf{MIT GR Course Problem Set 2}\\Solutions and Explanations}
\author{Peter Benjamin Zhu}
\date{}
\linespread{1.5}
\definecolor{shadecolor}{RGB}{241, 241, 255}
\newcounter{problemname}
\newenvironment{problem}{\begin{shaded}\stepcounter{problemname}\par\noindent\textbf{Problem\arabic{problemname}. }}{\end{shaded}\par}
\newenvironment{solution}{\par\noindent\textbf{Solution. }}{\par}
\newenvironment{note}{\par\noindent\textbf{Notes for Problem\arabic{problemname}. }}{\par}

\newcommand{\vc}[1]{\ensuremath{\overrightarrow{#1}}}
\renewcommand{\vb}[1]{\ensuremath{\mathbf{#1}}}
\newcommand{\p}{\partial}
\newcommand{\e}{\ensuremath{\mathrm{e}}}
\newcommand{\ii}{\ensuremath{\mathrm{i}}}

% 定义 \pv 命令(支持偏导数)
\DeclareDocumentCommand{\pv}{omg}{%
    \ensuremath{%
    \IfNoValueTF{#3}
    {\frac{\partial \IfNoValueTF{#1}{}{^{#1}}}{\partial #2\IfNoValueTF{#1}{}{^{#1}}}}
    {\frac{\partial \IfNoValueTF{#1}{}{^{#1}}#2}{\partial #3\IfNoValueTF{#1}{}{^{#1}}}}%
    }%
}

% % 定义可扩展的等号命令
% \makeatletter
% \newcommand{\xlongequal}[2][]{%
%   \ext@arrow 0055{\longequal}{#1}{#2}%
% }
% \def\longequal{\relbar\relbar\relbar\relbar\relbar}  % 定义等号的基础样式
% \makeatother

% 定义图文并排命令(可选)
\newcommand{\figwithtext}[5][0.35\linewidth]{
  % 参数1: 图形宽度(可选,默认 0.35\linewidth)
  % 参数2: 左侧文本
  % 参数3: 标题内容
  % 参数4: 标签名(用于 \label)
  % 参数5: 图形代码
  \begin{figure}[htbp]
    \begin{minipage}{0.6\linewidth}
      #2
    \end{minipage}%
    \hfill
    \begin{minipage}{#1}
      \centering
      #5
      \caption{#3}  % 动态标题
      \label{#4}    % 动态标签
    \end{minipage}
  \end{figure}
}

% 定义图文并排命令(支持外部图片)
\newcommand{\outerfigwithtext}[5][0.35\linewidth]{
  % 参数1: 图形宽度(可选,默认 0.35\linewidth)
  % 参数2: 左侧文本
  % 参数3: 标题内容
  % 参数4: 标签名(用于 \label)
  % 参数5: 图片路径(如 "figure.png")
  \begin{figure}[htbp]
    \begin{minipage}{0.6\linewidth}
      #2
    \end{minipage}%
    \hfill
    \begin{minipage}{#1}
      \centering
      \includegraphics[width=\linewidth]{#5}  % 插入图片,宽度自动适配
      \caption{#3}                            % 动态标题
      \label{#4}                              % 动态标签
    \end{minipage}
  \end{figure}
}


\begin{document}

\maketitle

\begin{problem}
    Show that the number density of dust measured by an observer whoes 4-velocity is $\vv{U}$ is given by $n=-\vv{N}\vdot\vv{U}$, where $\vv{N}$ is the matter current 4-velocity.
\end{problem}

\begin{solution}
    By definition we have the matter current 4-velocity $N^a=nU^a$, we have
    \begin{equation}
        \boxed{
        -N^aU_a=-nU^aU_a=n
        }
    \end{equation}
\end{solution}

% \begin{note}
%     这里是注记. 
% \end{note}

\begin{problem}
    Take the limit of continuity equation for $\abs{\vb{v}}\ll 1$ to get $\p n/\p t+\p\qty(nv^i)/\p x^i=0$
\end{problem}

\begin{solution}
    For $\abs{\vb{v}}\ll 1$, the Lorentz factor satisfies $\gamma \approx 1$. Therefore the 4-velocity can be expressed as $\vv{U}=(1,\vb{v})$.

    According to the continuity equation $\p_\alpha N^\alpha = 0$ and $N^\alpha = n U^\alpha$, we have
    \begin{equation}
            \begin{split}
                &\p_\alpha \qty(nU^\alpha)=0\\
                \Rightarrow &\p_t\qty(nU^t)+\p_i\qty(nU^i)=0\\
                \Rightarrow &\boxed{\pv{n}{t}+\pv{\qty(nv^i)}{x^i}=0}
            \end{split}
    \end{equation}
\end{solution}

\begin{problem}
    In an inertial frame $\mathscr{O}$, calculate the components of the stress-energy tensors of the following systems:

    (a) A group of prticles all moving with the same 3-velocity $\vb{v}=\beta\vv*{e}{x}$ as seen in $\mathscr{O}$. Let the rest-mass desnity of these particles be $\rho_0$, as measured in their own rest frame. Assume a sufficiently high density of particles to enable treating them as a continuum.

    (b) A ring of $N$ similar particles of rest mass $m$ rotating counter-clockwise in the x-y plane about the origin of $\mathscr{O}$, at a radius $a$ from this point, with an angular velocity $\omega$. The ring is a torus of circular cross-section $\delta a \ll a$, within which the particles are uniformly distributed with a high enough density for the continuum approximation to apply. Do not include the stress-energy of whatever forces keep them in orbit. Part of this calculation will relate $\rho_0$ of part (a) to $N$, $a$, $\delta a$, and $\omega$.

    (c) Two such rings of particles, one rotating clockwise and the other counter-clockwise, at the same radius $a$. The particles do not collide or otherwise interact in any way.
\end{problem}

\begin{solution}

    (a) First, in the rest frame of the group of these particles, the stress-energy tensor $T^{\mu\nu}$ assumes the simplified form $T^{\mu\nu}=\rho_0 U^\mu U^\nu$, or in matrix:
    \[
    \mqty(\dmat{\rho_0, 0, 0, 0})
    \]
    Second, given these particles move in the same 3-velocity $\vb{v}=\beta\vv*{e}{x}$ as seen in $\mathscr{O}$, their 4-velocity in this frame becomes $\vv{U'}=(\gamma, \gamma\beta, 0, 0)$. Consequently, the stress-energy tensor in frame $\mathscr{O}$ assumes the form:
    \begin{equation}
        \boxed{
        T'^{\mu\nu}=\rho_0U'^\mu U'^\nu=\gamma^2\rho_0 \mqty(1&\beta&0&0\\\beta&\beta^2&0&0\\0&0&0&0\\0&0&0&0)\label{eq:stressenergy}
        }
    \end{equation}
    
    (b) First, given the torus volume $V=2\pi a \pi (\delta a)^2$, the rest-mass density $\rho_0$ is determined as:
    \begin{equation}
        \rho_0 = \frac{Nm}{2\pi^2a(\delta a)^2}\label{eq:massdensity}
    \end{equation}
    Next, for particles undergoing counter-clockwise rotation with angular velocity $\omega$, the corresponding 4-velocity components at the angle $\phi$ are derived via Cartesian coordinates:
    \begin{equation}
        \vv{U}=\gamma\qty(1, -a\omega\sin(\phi), a\omega\cos(\phi), 0), \quad \gamma=\qty(1-a^2\omega^2)^{-\frac{1}{2}}\label{eq:4velocity}
    \end{equation}
    Finally, combining the system's symetry and the strss-energy tensor definition $T^{\mu\nu}=\rho_0U^\mu U^\nu$, angular averaging over $\phi$ yields:
    \begin{equation}
        \begin{split}\label{eq:averageTs}
            &\expval{T^{00}}=\gamma^2\rho_0\\
            &\expval{T^{11}}=\gamma^2\rho_0\vdot\frac{1}{2}a^2\omega^2\\
            &\expval{T^{22}}=\gamma^2\rho_0\vdot\frac{1}{2}a^2\omega^2\\
            &\expval{T^{12}}=\expval{T^{21}}=\expval{\sin\phi\cos\phi}=0
        \end{split}
    \end{equation}
    Combining \cref{eq:massdensity}, \cref{eq:4velocity} and \cref{eq:averageTs} to get the final stress-energy tensor assuming the matrix representation:
    \begin{equation}
        \boxed{
        T^{\mu\nu}=\gamma^2\rho_0\mqty(\dmat{1, \frac{1}{2}a^2\omega^2, \frac{1}{2}a^2\omega^2, 0})\label{eq:singleT}
        }
    \end{equation}

    (c) Since the two rings are counter-rotating and the constituent particles exhibit no interactions, the stress-energy tensor for each ring independently assumes the form specified in \cref{eq:singleT}. The composite system's stress-energy tensor is therefore given by linear superposition:
    \begin{equation}
        \boxed{
            T^{\mu\nu}=2\gamma^2\rho_0\mqty(\dmat{1, \frac{1}{2}a^2\omega^2, \frac{1}{2}a^2\omega^2, 0})
        }
    \end{equation}
\end{solution}

\begin{note}

    In problem (a), the stress-energy tensor $T'^{\mu\nu}$ can be derived via an alternative approach distinct from \cref{eq:stressenergy}:
    \begin{equation}
        T'^{\mu\nu}=\pdv{x'^\mu}{x^\sigma}\pdv{x'^\nu}{x^\rho}T^{\sigma\rho}=\pdv{x'^\mu}{x^0}\pdv{x'^\nu}{x^0}T^{00}\label{eq:altstressenergy}
    \end{equation}
    Evidently, the application of tensor transformation formula in \cref{eq:altstressenergy} yields identical result.
\end{note}

\begin{problem}
    Use the identity $\p_\nu T^{\mu\nu}=0$ to prove the following results for a bounded system (i.e., a system for which $T^{\mu\nu}=0$ beyond some bounded region of space):

    (a) $\p_t\int T^{0\alpha}\dd[3]{x} = 0$. This expresses conservation of energy and momentum.

    (b) $\p^2_t\int T^{00}x^i x^j\dd[3]{x}=2\int T^{ij}\dd[3]{x}$. This result is a version of the virial theorem; it will come in quite handy when we derive the quadrupole formula for gravitational radiation.

    (c) $\p^2_t\int T^{00}\qty(x^i x_i)^2\dd[3]{x}=4\int T^i_i x^j x_j\dd[3]{x}+8\int T^{ij}x_i x_j\dd[3]{x}$. No pithy wisdom for this one.
\end{problem}

\begin{solution}
    Utilizing the identity $\p_\nu T^{\mu\nu}=0$, we derive the following equations:
    \begin{equation}
        \begin{cases}
            \p_t T^{00}+\p_k T^{0k}=0, \quad \mu=0\\
            \p_t T^{0i}+\p_k T^{ik}=0, \quad \mu=i
        \end{cases}
        \qquad
        (i, j, k =1, 2, 3)\label{eq:partialTdifferentcase}
    \end{equation}
    Later we will use \cref{eq:partialTdifferentcase} for multiple times, so it's best to put these relations before the steps of the solutions.

    (a) For the $\mu=0$ component in \cref{eq:partialTdifferentcase}, integration to get
    \begin{equation}
        \p_t\int T^{00}\dd[3]{x}=-\int\p_k T^{0k}\dd[3]{x}
    \end{equation}
    Applying Gauss's law we get $\int \p_k T^{0k}\dd[3]{x}=\oint T^{0k}\dd{S_k}$. Given that it is a bounded system, $T^{0k}$ vanishes at spatial infinity, leading to:
    \begin{equation}
        \boxed{
        \p_t \int T^{00}\dd[3]{x} = 0 \quad (\text{Conservation of Energy})
        }
    \end{equation}
    Similarly, for $\mu = i$ we have:
    \begin{equation}
        \boxed{
        \p_t \int T^{0i}\dd[3]{x} = 0 \quad (\text{Conservation of Momentum})
        }
    \end{equation}
    This completes the proof of $\p_t \int T^{0\alpha}\dd[3]{x}=0\quad (\alpha = 0, 1, 2, 3)$.

    (b) According to \cref{eq:partialTdifferentcase} we have:
    \begin{equation*}
        \p_t \int T^{00}x^i x^j \dd[3]{x}=-\p_k \int T^{0k}x^i x^j\dd[3]{x}
    \end{equation*}
    Applying integration by parts yields:
    \begin{equation*}
        \begin{split}
            &-\int \p_k T^{0k}x^i x^j\dd[3]{x}=\int T^{0k}\p_k \qty(x^i x^j)\dd[3]{x}-0\\
            &=\int T^{0k}\qty(x^j\delta^i_k + x^i\delta^j_k)\dd[3]{x}=\int\qty(T^{0i}x^j+T^{0j}x^i)\dd[3]{x}
        \end{split}
    \end{equation*}
    Subsequently taking the time derivative once and combining \cref{eq:partialTdifferentcase} when $\mu=i$ provides:
    \begin{equation*}
        \begin{split}
            &\p_t\int \qty(T^{0i}x^j+T^{0j}x^i)\dd[3]{x}=\int\qty(\p_k T^{ik}x^j+\p_k T^{jk}x^i)\dd[3]{x}\\
            &=\int \qty(T^{ik}\delta^j_k+T^{jk}\delta^i_k)\dd[3]{x}=2\int T^{ij}\dd[3]{x}
        \end{split}
    \end{equation*}
    Thus, we prove the identity $\boxed{\p^2_t\int T^{00}x^i x^j\dd[3]{x}=2\int T^{ij}\dd[3]{x}}$.

    (c) According to \cref{eq:partialTdifferentcase} we have:
    \begin{equation*}
        \p_t \int T^{00}\qty(x^i x_i)^2 \dd[3]{x}=-\p_k \int T^{0k}\qty(x^i x_i)^2\dd[3]{x}
    \end{equation*}
    Applying integration by parts yields:
    \begin{equation*}
        \begin{split}
            &-\p_k \int T^{0k}\qty(x^i x_i)^2\dd[3]{x}=\int T^{0k}\p_k\qty(x^i x_i)^2\dd[3]{x}\\
            &=\int T^{0k}\vdot 2(x^i x_i) (x^i\delta_{ik}+x_i\delta^i_K)\dd[3]{x}=\int T^{0k}\vdot 2(x^i x_i) \vdot 2x_k \dd[3]{x}\\
            &=4\int T^{0k}(x^i x_i)x_k\dd[3]{x}
        \end{split}
    \end{equation*}
    Subsequently taking the time derivative once and combining \cref{eq:partialTdifferentcase} when $\mu=i$ provides:
    \begin{equation*}
        \begin{split}
            &4\int \p_t T^{0k}(x^i x_i)x_k\dd[3]{x}=-4\int \p_j T^{jk}x_k(x^i x_i)\dd[3]{x}\\
            &=4\int T^{jk}\p_j\qty[x_k(x^i x_i)]\dd[3]{x}\\
            &=4\int T^{jk}x^i x_i\delta_{jk}\dd[3]{x}+8\int T^{jk}x_k x_j\dd[3]{x}\\
            &=4\int T^j_j x^i x_i\dd[3]{x}+8\int T^{jk}x_j x_k\dd[3]{x}
        \end{split}
    \end{equation*}
    Regroup the index to get the identity $\p^2_t\int T^{00}\qty(x^i x_i)^2\dd[3]{x}=4\int T^i_i x^j x_j\dd[3]{x}+8\int T^{ij}x_i x_j\dd[3]{x}$.
\end{solution}

\begin{problem}

    The vector potential $\vv{A} \doteq \qty(A^0, \vb{A})$ generates the electromagnetic field tensor via
    \[
    F_{\mu\nu}=\p_\mu A_\nu - \p_\nu A_\mu
    \]

    (a) Show that the electric and magnetic fields in a special Lorentz frame are given by
    \begin{align*}
        \vb{B} & = \curl{\vb{A}}\\
        \vb{E} & = -\pdv{\vb{A}}{t}-\grad{A^0}.
    \end{align*}

    (b) Show that the Maxwell's equations hold if and only if
    \[
    \p_\mu \p^\mu A^\alpha - \p^\alpha \p_\mu A^\mu = -4\pi J^\alpha
    \]

    (c) Show that a gauge transformation of the form
    \[
    A^{\text{new}}_\mu = A^{\text{old}}_\mu + \p_\mu \phi
    \]
    leaves the field tensor unchanged.

    (d) Show that one can adjust the gauge so that
    \[
    \p_\mu A^\mu = 0
    \]
    Show that Maxwell's equations take on a particularly simple form with this gauge choice. Use the operator $\Box \equiv \p_\mu \p^\mu$ to simplify your result.
\end{problem}

\begin{solution}
    The components of electromagnetic field tensor in a particular reference frame assume the matrix form:
    \begin{equation}
        \qty(F_{\mu\nu})=\mqty[0 & -E_1 & -E_2 & -E_3 \\ E_1 & 0 & B_3 & -B_2 \\ E_2 & -B_3 & 0 & B_1 \\ E_3 & B_2 & -B_1 & 0]\label{eq:emtensorcomponents}
    \end{equation}

    (a) By utilizing the Levi-Civita symbol $\epsilon_{ijk}$, the curl of the vector \vb{A} can be reformulated in the following tensorial notation:
    \begin{equation*}
        \curl{\vb{A}}=\epsilon^{ijk}\p_i A_j
    \end{equation*}
    By definition, the electric and magnetic field vectors can be expressed by $F_{\mu\nu}$:
    \begin{align}
        E_\mu =& F_{\mu 0}\label{eq:Evector}\\
        B_\mu =& \frac{1}{2}\epsilon_{0\mu\nu\sigma}F^{\nu\sigma}\label{eq:Bvector}
    \end{align}
    where $\mu, \nu, \sigma = 1, 2, 3$. Since $F_{\mu\nu}=\p_\mu A_\nu - \p_\nu A_\mu$ and $F^{ij}=F_{ij}$, combining \cref{eq:Evector} and \cref{eq:Bvector} yields:
    \begin{align}
        E_\mu = & \p_\mu A_0 - \p_0 A_\mu\\
        B_\mu = & \frac{1}{2}\epsilon_{0\mu\nu\sigma}\qty(\p_\nu A_\sigma-\p_\sigma A_\nu)=\epsilon_{0\mu\nu\sigma}\p_{[\nu}A_{\sigma]}=\epsilon_{0\mu\nu\sigma}\p_\nu A_\sigma
    \end{align}
    Because of $A_0 = -A^0$, the electric and magnetic field vectors assume the more familiar form:
    \begin{align}
        \boxed{
            \begin{aligned}
                E_\mu =& -\p_\mu A^0 - \p_0 A_\mu \doteq -\pdv{\vb{A}}{t}-\grad{A^0}\\
            B_\mu =& \epsilon_{0\mu\nu\sigma}\p_\nu A_\sigma \doteq \curl{\vb{A}}
            \end{aligned}
        }
    \end{align}

    (b) Starting from the 4-dimensional Maxwell's equations
    \begin{equation*}
        \p^\mu F_{\mu\nu}=-4\pi J_\nu
    \end{equation*}
    we raise indices via the Minkowski metric tensor $\eta^{\mu\nu}$ to obtain the countravariant form:
    \begin{equation*}
        \p_\mu F^{\mu\nu}=-4\pi J^\nu
    \end{equation*}
    Substituting the field tensor definition $F_{\mu\nu}=\p_\mu A_\nu - \p_\nu A_\mu$ yields:
    \begin{equation*}
        \p_\mu\qty(\p^\mu A^\nu - \p^\nu A^\mu)=-4\pi J^\nu
    \end{equation*}
    Performing the index substitution $\nu\to\alpha$, we arrive at the final form:
    \begin{equation}
        \boxed{
            \p_\mu\p^\mu A^\alpha - \p^\alpha\p_\mu A^\mu = 4\pi J^\alpha
        }\label{eq:4vectorpotential}
    \end{equation}

    (c) We can easily prove this gauge transformation formula by substitution:
    \begin{equation}
        \begin{split}
            &\boxed{F^{\text{new}}_{\mu\nu}=\p_\mu A^\text{new}_\nu - \p_\nu A^\text{new}_\mu} = \p_\mu \qty(A^\text{old}_\nu+\p_\nu \phi) - \p_\nu \qty(A^\text{old}_\mu+\p_\mu \phi)\\
            &\p_\mu A^\text{old}_\nu+\p_\mu\p_\nu \phi - \p_\nu A^\text{old}_\mu+\p_\nu\p_\mu \phi = \boxed{\p_\mu A^\text{old}_\nu - \p_\nu A^\text{old}_\mu = F^\text{old}_{\mu\nu}}
        \end{split}
    \end{equation}

    (d) Under the gauge transformation $A^\mu = A'^\mu + \p^\mu \phi$, the divergence of the 4-potential transforms as:
    \begin{equation*}
        \p_\mu A^\mu = \p_\mu A'^\mu + \p_\mu \p^\mu \phi
    \end{equation*}
    Employing the Lorentz gauge condition $\p_\mu A^\mu = 0$, we obtain a wave equation for the gauge function $\phi$:
    \begin{equation*}
        \p_\mu \p^\mu \phi = -\p_\mu A'^\mu
    \end{equation*}
    Provided $A'$ satisfies the integrability condition $\p_\nu\qty(\p_\mu A'^\mu)=\p_\mu\qty(\p_\nu A'^\mu)$, there always exists a non-trivial solution $\phi$ for given initial and boundary conditions. This guarantees the existence of a gauge function $\phi$ that enforces $\p_\mu A^\mu=0$. Consequently, \cref{eq:4vectorpotential} simplifies to the form:
    \begin{equation}
        \boxed{
            \Box A^\alpha = -4\pi J^\alpha \quad \text{with}\quad \Box \equiv \p_\mu \p^\mu
        }
    \end{equation}
\end{solution}

\begin{problem}
    An astronaut has acceleration $g$ in the $x$ direction (in other words, the magnitude of his 4-acceleration, $\sqrt{\vv{a}\vdot\vv{a}}$ is $g$). This astronaut assigns coordinates $\qty(\bar{t}, \bar{x}, \bar{y}, \bar{z})$ to spacetime as follows:
    
    First, the astronaut defines spatial coordinates to be $\qty(\bar{x}, \bar{y}, \bar{z})$, and sets the time coordinates $\bar{t}$ to be his own proper time.

    Second, at $\bar{t}=0$, the astronaut assigns $\qty(\bar{x}, \bar{y}, \bar{z})$ to coincide with the Euclidean coordinates $(x, y, z)$ of inertial reference frame that momentarily coincides with his motion. (In other words, though the astronaut is not inertial --- he is accelerating --- there is an inertial frame that, at $\bar{t}=0$, is momentarily at rest with respect to him. This is the frame used to assign $\qty(\bar{x}, \bar{y}, \bar{z})$ at $\bar{t}=0$) Observers who remain at fixed values of the spatial coordinates $\qty(\bar{x}, \bar{y}, \bar{z})$ are called coordinate-stationary observers (CSOs). Note that proper time for these observers is not neccessarily $\bar{t}$! ---- we cannot assume that the CSOs' clocks remain synchronized with the clocks of the astronaut. Assume that some function $A$ converts between coordinate time $\bar{t}$ and proper time at the location of CSO:
    \[
    A=\dv{\bar{t}}{\tau}
    \]
    The function $A$ is evaluated at a CSO's location and thus can in principle depend on all four coordinates $\bar{t},\ \bar{x},\ \bar{y},\ \bar{z}$.

    Finally, the astronaut requires that the worldlines of CSOs must be orthogonal to the hypersurface $\bar{t} = \text{constant}$, and that for each $\bar{t}$ there exists an inertial frame, momentarily at rest with respect to the astronaut, in which all events with $\bar{t}=\text{constant}$ are simultaneous.

    It is easy to see that $\bar{y}=y$ and $\bar{z}=z$; henceforth we drop this coordinates from the problem.

    (a) What is the 4-velocity of the astronaut, as a function of $\bar{t}$, in the initial inertial frame [the frame that uses coordinates $(t,\ x,\ y,\ z)$]?(Hint: by considering the conditions on $\vv{u}\vdot\vv{u}, \vv{u}\vdot\vv{a}$, and $\vv{a}\vdot\vv{a}$, you should be able to find simple forms for $u^t$ and $u^x$.)

    (b) Imagine that each coordinate-stationary observer carries a clock. What is the 4-velocity of each clock in the intial inertial frame?

    (c) Explain why $A\qty(\bar{x}, \bar{t})$ cannot depend on time. In other words, why can we put $A\qty(\bar{x}, \bar{t})=A(\bar{x})$?(Hint: consider the coordinate system that a different CSO may set up.)

    (d) Find an explicit solution for the coordinate transformation $x\qty(\bar{t}, \bar{x})$ and $t\qty(\bar{t}, \bar{x})$.

    (e) Show that the line element $ds^2 = -dt^2 + dx^2 = -\qty(1+g\bar{x})^2d\bar{t}^2 + d\bar{x}^2$.
\end{problem}

\begin{solution}
    (a) Since the astronaut's 4-velocity possesses only a spatial component along x direction, we may model their motion in a 1+1-dimensional spacetime. Let the 4-velocity be $\vv{U}=\qty(u_t, u_x)$. The normalization condition of 4-velocity gives:
    \begin{equation}
        U^\alpha U_\alpha = -u_t^2 + u_x^2 = -1\label{eq:as4velocity}
    \end{equation}
    Give the astronaut's 4-acceleration magnitude $g$, the 4-acceleration satisfies:
    \begin{equation}
        A^\alpha A_\alpha = -a_t^2 + a_x^2 = g^2\label{eq:as4acceleration}
    \end{equation}
    Orthogonality between 4-velocity and 4-acceleration requires:
    \begin{equation}
        A^\alpha U_\alpha = -a_t u_t + a_x u_x = 0\label{eq:asorthogonal}
    \end{equation}
    \textbf{Boundary Condition:}
    \begin{itemize}
        \item At $\bar{t}=0$: $u_t=0$, $u_x=0$;
        \item For $\bar{t}>0$: $u_x>0$
    \end{itemize}
    Combining the Boundary Condition and \cref{eq:as4velocity} - \cref{eq:asorthogonal} we get the solution:
    \begin{align}
        \boxed{
            \begin{aligned}
                \vv{U} =& \qty(\cosh (g\bar{t}), \sinh (g\bar{t}))\\
                \vv{A} =& \qty(g\sinh (g\bar{t}), g\cosh (g\bar{t}))
            \end{aligned}
        }\label{eq:asmotion}
    \end{align}

    Problems (b)-(e) fundamentally constitute different facets of a unified physical scenario. The subsequent analysis will synthesize these components through an integrated treatment, deliberately omitting distinct problem identifiers to emphasize their intrinsic coherence.

    First, let's explain the reason why $A(\bar{x}, \bar{t})$ cannot depend on time, as to say $A(\bar{x}, \bar{t})=A(\bar{x})$. 
    \begin{enumerate}
        \item \textbf{Coordinate Consistency:} If $A$ depends on $\bar{t}$, coordinate-stationary observers (CSOs) at different times would establish divergent time synchronization standards, thereby precluding the cunstruction of a globally consistent coordinate system.
        \item \textbf{Symmetry Consideration:} While the astronaut's fixed acceleration direction preserves spatial translational symmetry, explicit time dependence in the metric components breaks this symmetry.
        \item \textbf{Orthogonality Constraints:} The requierment that the CSOs' worldlines remain orthogonal to constant-$\bar{t}$ hypersurfaces enforces $A$ to depend exclusively on spatial coordinates.
    \end{enumerate}
    Second, integrate \cref{eq:asmotion} to get astronaut's equations of motion:
    \begin{align}
        t =& \frac{1}{g}\sinh\qty(g\bar{t})\label{eq:asteom}\\
        x =& \frac{1}{g}\cosh\qty(g\bar{t})-\frac{1}{g}\label{eq:asxeom}
    \end{align}
    Assuming that CSOs' equations of motion are similar to \cref{eq:asteom} and \cref{eq:asxeom}, but are parameterized by $\bar{x}$ in the following way:
    \begin{align}
        t =& T\qty(\bar{x})\sinh\qty(g\bar{t})\label{eq:csointeom}\\
        x =& X\qty(\bar{x})\cosh\qty(g\bar{t})+Y\qty(\bar{x})\label{eq:csoinxeom}
    \end{align}
    Obviously, CSOs' 4-velocity components assume the form:
    \begin{equation}
        \begin{split}
            u_t =& gAT\cosh\qty(g\bar{t})\\
            u_x =& gAX\sinh\qty(g\bar{t})
        \end{split}
        \quad \text{with} \quad A=\dv{\bar{t}}{\tau}
    \end{equation}
    According to \cref{eq:as4velocity}, we obtain:
    \begin{equation}
        \begin{split}
            &g^2A^2(T^2\cosh^2\qty(g\bar{t})-X^2\sinh^2\qty(g\bar{t}))=1\\
            &\Rightarrow\boxed{
            \begin{cases}
                T=&X\\
                X=&\frac{1}{gA}
            \end{cases}}
        \end{split}\label{eq:asXTequal}
    \end{equation}
    The reason why \cref{eq:asXTequal} can be solved like this is that $A$, $T$, $X$ are all independent to $\bar{t}$. Now, let's what will happen when we try to obtain the expression of the line element.
    \begin{align}
        -dt^2+dx^2&=-g^2T^2\cosh^2(g\bar{t})d\bar{t}^2-\qty(\dv{T}{\bar{x}})^2\sinh(g\bar{t})d\bar{x}^2\nonumber\\
                  &-2gT\dv{T}{\bar{x}}\cosh(g\bar{t})\sinh(g\bar{t})d\bar{t}d\bar{x}+g^2X^2\sinh^2(g\bar{t})d\bar{t}^2\nonumber\\
                  &+\qty(\dv{X}{\bar{x}})^2\cosh(g\bar{t})d\bar{x}^2+\qty(\dv{Y}{\bar{x}})^2d\bar{x}^2+2\dv{X}{\bar{x}}\dv{Y}{\bar{x}}\cosh(g\bar{t})d\bar{x}^2\nonumber\\
                  &=2gX\dv{X}{\bar{x}}\cosh(g\bar{t})\sinh(g\bar{t})d\bar{t}d\bar{x}+2gX\dv{Y}{y}\sinh(g\bar{t})d\bar{t}d\bar{x}\nonumber\\
                  &=-\qty(g^2T^2\cosh^2(g\bar{t})-g^2X^2\sinh(g\bar{t}))d\bar{t}^2-\qty(\dv{T}{\bar{x}})^2\sinh^2(g\bar{t})d\bar{x}^2\nonumber\\
                  &+\qty(\dv{X}{\bar{x}})\cosh^2(g\bar{t})d\bar{x}^2+\qty(\dv{Y}{\bar{x}})^2d\bar{x}^2+2\dv{X}{\bar{{x}}}\dv{Y}{\bar{x}}\cosh(g\bar{t})d\bar{x}^2\nonumber\\
                  &+2gX\dv{X}{\bar{x}}\cosh(g\bar{t})\sinh(g\bar{t})d\bar{t}d\bar{x}+2gX\dv{Y}{\bar{x}}\sinh(g\bar{t})d\bar{t}d\bar{x}\nonumber\\
                  &-2gT\dv{T}{\bar{x}}\cosh(g\bar{t})\sinh(g\bar{t})d\bar{t}d\bar{x}
    \end{align}
    Combining \cref{eq:asXTequal} and according to the \textbf{Symmetry Consideration} part, the metric tensor exhibits no explicit dependence on the coordinate $\bar{t}$ and contains no temporal-spatial cross terms, which means:
    \begin{equation}
        \begin{cases}
            \frac{X}{\cosh(g\bar{t})}\dv{Y}{\bar{x}}=0\\
            -dt^2+dx^2=-g^2X^2d\bar{t}^2+\qty(\dv{X}{\bar{x}})^2d\bar{x}^2
        \end{cases}
    \end{equation}
    It therefore follows that $Y$ must be a constant. To determine its explicit form, considering the following boundary conditions:
    \begin{itemize}
        \item \textbf{Initial Inertial Frame Alignment:} At $\bar{t}=0$, the CSOs' coordinates coincide with the initial inertial coordinate $\bar{x}$ yielding:
        \begin{equation}
            X+Y=\bar{x}\label{eq:XY}
        \end{equation}
        \item \textbf{Observer Self-Consistency Condition:} When $\bar{x}=0$ (the astronaut serves as his own CSO), the coordinate must reduce to the astronaut's known equations of motion:
        \begin{equation}
            -Y\cosh(g\bar{t})+Y=\frac{1}{g}\cosh(g\bar{t})-\frac{1}{g}\label{eq:Y}
        \end{equation}
    \end{itemize}
    Combining \cref{eq:XY} and \cref{eq:Y} to obtain:
    \begin{align}
        \boxed{
            \begin{aligned}
                X=&\bar{x}+\frac{1}{g}\\
                Y=&-\frac{1}{g}
            \end{aligned}
        }
    \end{align}
    By \cref{eq:asXTequal} we obtain:
    \begin{equation}
        \boxed{A=\frac{1}{1+g\bar{x}}}
    \end{equation}
    Now we can write the completed version of CSOs equations of motion:
    \begin{align}
        \boxed{
            \begin{aligned}
                t=&\qty(\bar{x}+\frac{1}{g})\sinh(g\bar{t})\\
                x=&\qty(\bar{x}+\frac{1}{g})\cosh(g\bar{t})-\frac{1}{g}\\
                u_t=&g\qty(\bar{x}+\frac{1}{g})\frac{1}{1+g\bar{x}}\cosh(g\bar{t})=\cosh(g\bar{t})\\
                u_x=&g\qty(\bar{x}+\frac{1}{g})\frac{1}{q+g\bar{x}}\sinh(g\bar{t})=\sinh(g\bar{t})
            \end{aligned}
        }
    \end{align}
    Henceforth, the line element assumes the simplified form:
    \begin{equation}
        \boxed{ds^2=-dt^2+dx^2=-(1+g^2\bar{x})^2d\bar{t}^2+d\bar{x}^2}
    \end{equation}
    This spacetime is called the Rindler spacetime and the metric is called Rindler metric.
\end{solution}

\end{document}